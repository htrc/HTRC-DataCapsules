\documentclass{acm_proc_article-sp}

\usepackage{epstopdf}

\begin{document}

\title{Improving the Usability of Remote Access to Sensitive Data}

\author{
Draft Copy
}

%\numberofauthors{2} 
%
%\author{
%\alignauthor Alexander Crowell \\
%       \affaddr{Computer Science and Engineering Division}\\
%       \affaddr{University of Mighigan}\\
%       \affaddr{Ann Arbor, MI, USA}\\
%       \email{crowella@umich.edu}
%\alignauthor Atul Prakash \\
%       \affaddr{Computer Science and Engineering Division}\\
%       \affaddr{University of Mighigan}\\
%       \affaddr{Ann Arbor, MI, USA}\\
%       \email{aprakash@umich.edu}
%}

\maketitle
\begin{abstract}

As we enter an era of ``big data", where analysis of large-scale data is
revealing important insights into a variety of fields, there is an ever greater
demand for access to new data wherein the potential for new insights may lie.
However, in many cases, this need for access conflicts with the desire to
protect data privacy.  Indeed, many types of important data present just such a
dilemma, including copyrighted data, personal information of individuals, and
data containing state or corporate secrets.

In this paper, we propose Data Capsules, a system that is designed to enable
access to sensitive data for analysis by trusted remote users, while
maintaining reasonable guarantees of data security.  Data Capsules uses
virtualization to provide remote users with a privileged, but secure
environment into which they can bring arbitrary, and even potentially
malicious, software or data in order to analyze sensitive data, while
minimizing the available channels for data leaks.  Our early implementation
realizes much of this protection, providing a basic framework for secure analysis of data that addresses many aspects of network, storage, and covert channel security.

\end{abstract}

\category{D.4.6}{Operating Systems}{Security and Protection}
\category{H.3.4}{Systems and Software}{Distributed Systems}[cloud computing, data capsules]

\keywords{data capsules, data privacy}

\section{Introduction}

In the age of the Internet, new data is being created at incredible rates
\cite{digital-universe}.  Along with these waves of new data, new tools have
in turn been created to help analyze and gain insight from it. Unfortunately,
access to data is not always a simple matter.  Some types of data may be
sensitive, including data protected by copyright, data representing the personal
information of individuals, as well as any data that may contain secrets that
its owner does not wish to have revealed, for example corporate or state secrets.

One good example of a system for access to sensitive data is the data for the
United States Census.  The federal government established Census Research Data
Centers (RDCs) to enable researchers access to unpublished Census data for the
purpose of conducting research that could provide beneficial guidance to public
policy.  Once granted access, a researcher must visit one of the CRDCs in
person, and will then be permitted to use preallocated machines with preloaded
software in order to conduct their research.  Upon completion of their research,
their results are human-audited to protect confidentiality, and then released.
While this system provides access to Census data, and seems robust to data
leaks, it is also highly restrictive, requiring researchers to relocate and use
foreign hardware to conduct research.

More recently, the Centers for Medicare and Medicaid Services began providing
Medicare and Medicaid data using the CMS Virtual Research Data Center (VRDC) in
2013.  VRDC provides a more flexible model by allowing researchers to remotely
access desktop environments provided by a cloud infrastructure.  This model
relieves researchers of the need to travel to a physical location, and also
avoids the security risk of researchers having to receive and safeguard physical
copies of the data, while they are conducting their research.

However, even this remote access model is still restrictive, in that researchers
are largely limited to the data analysis tools and software that are made
available to them in their virtual environments, and any new data or software
one wishes to upload to a research environment must first undergo a review
process. To solve this substantial inconvenience, we propose a new system which
we call Data Capsules.  While, to an end user, a Data Capsules environment will
minimally differ from the type of environment provided by the CMS Virtual
Research Data Center, its design grants end users privileged access to install
software of their choice --- even down to the virtualization-compatible
operating system of their choice --- without requiring a review process and
without any compromise to the security of the data being provided.

Data Capsules makes use of virtualization as a key element in achieving this
security and usability goal.  Under the Data Capsules use model, a researcher
would be allocated a virtual machine (henceforth abbreviated as VM, or guest
VM), which they could then prepare and maintain with whatever software and
underlying operating system is best suited to the intended purpose.  Each Data
Capsules VM is endowed with two modes of operation.  In \emph{maintenance} mode,
the researcher is endowed with full Internet access, which enables them to
easily configure and maintain the software environment almost as if they were
using a their own personal machine; however, in maintenance mode, the researcher
does not have access to the sensitive data.  In order to gain access to this
data, the user must switch their VM to the second mode, which we call
\emph{security} mode.  In security mode, the user's VM state is saved, and
execution continues on an alternate machine that is firewalled to prevent
Internet access, but which does have access to the sensitive data.

In this manner, once a user has configured their system and is prepared to
run their analysis on the data, they simply switch to security mode and start
the analysis.  Any results obtained from this analysis can then be stored in a
designated \emph{secure storage volume}, which is only attached to the VM in
secure mode.  Since this result data is only accessible in secure mode, it
follows that data can only be released from the virtual environment when in
secure mode.

When a user wishes to further change their software environment in order to
continue data analysis, they may then switch back to maintenance mode, after
which the secure storage volume will detatch, and all VM state since the
previous switch to secure mode is discarded, with the VM continuing to run from
the point it left off at when last in maintenance mode.  In this way, every
trace of the operations performed on the sensitive data is erased and forgotten.

Furthermore, assuming any sensitive data is read-only, and is properly backed
up, Data Capsules functionality means that there is no need to audit installed
software for malware, as even infected software would be unable to exfiltrate
the data from the secure mode environment.

In the following sections, we discuss related work, elaborate on the design and implementation of the Data Capsules system, and discuss the current state of the project and potential directions for future work.

\section{Related Work}


\section{Design}

We now discuss the design of the Data Capsules system.  This includes the
functional requirements for the two modes of operation and a discussion of how
the system avoids various channels for leaking data.

\subsection{Security Guarantees and Threat Model}

The goal of the Data Capsules system is to provide a usable environment for data
analysis that also provides a reasonably strong (as we elaborate on below)
guarantee of data privacy from parties other than the end user to whom a VM
belongs.  Data Capsules is not designed to provide any protection to the
integrity of sensitive data.  This is assumed trivial, as the sensitive data is
presumably unchanging, so that making it read-only and creating appropriate
backup copies should be a sufficient practice to ensure the data's integrity.

The end user is assumed to be largely trustworthy, since they have presumably
undergone a vetting process before receiving permission to analyze the sensitive
data.  However, we do recognize the possibility that others could gain temporary
access to a VM, for example by using a researcher's machine after they
momentarily step away.  If the device through which an end user accesses their
remote VM becomes compromised, this could also open the possibility of long-term
access to the VM.  While the current implementation does not fully address
either of these possibilities, we discuss the paths we are investigating in
Section \ref{sec:Discussion}. 

Within a user's VM, we make no assumptions about the software environment and
treat everything as untrusted, down to the operating system.  Data Capsules is
designed so that malicious software may succeed in accessing the sensitive data,
but will be incapable of reliably exfiltrating it from the virtual machine.
Under this design, we consider the following potential channels for data leaks:

\textbf{Storage Channels} It should not be possible for data that is produced
as the result of an operation whose input includes sensitive data to be stored
in such a way that it is accessible outside of secure mode.

\textbf{Network Channels} When a VM has access to sensitive data, e.g. when it
is in secure mode, it should not be able to access any other machine over the
network, with the exception of machines that are serving the sensitive data.

\textbf{Virtual Machine Access Channels} End users must access their VMs
through some network channel that connects a physical device (e.g. the user's
laptop) to the Data Capsules infrastructure.  This could be a remote terminal
connection such as SSH, or a remote desktop connection such as VNC or RDP.  In
both cases, steps must be taken to limit or eliminate channel bandwidth in order
to protect against the consequences of short and long-term compromises.

\textbf{Intramachine Covert Channels} If there are multiple VMs sharing a
single physical machine, Data Capsules must ensure that it is not possible for
one VM running in secure mode (and hence with access to sensitive data) to then
leak that data to a separate coresident VM running in maintenance mode, which
could then easily relay that data through any of the aforementioned channels.

\textbf{Intermachine Covert Channels} If there are multiple physical machines
serving VMs, there may be unexpected covert channels between them that could
allow similar data leaks from secure to maintenance VMs.  Although such attacks
are likely somewhat difficult to implement and unlikely to have very high
bandwidth, steps should be taken to minimize any such known channels to the
highest degree possible without substantially impacting performance.

The virtual machine infrastructure that underlies the Data Capsules system is
trusted and assumed to be well implemented.  While such an assumption may be
somewhat controversial, we note its use in prior work~\cite{capsules}, and also
note the existence of microkernels such as seL4 and OKL4~\cite{sel4, OKL4},
which serve as a proof of concept that it is possible to create a lightweight
hypervisor with a reasonable degree of trustworthiness.  However, the
development of such a hypervisor lies outside the scope of this paper.

\subsection{Design Overview}

Data Capsules makes heavy use of virtualization.  For each system user, there
are one or more corresponding VMs.  Multiple VMs per user may be useful in
situations where there is a large repository of data that is separated into
multiple corpora each with their own access permissions.  In this case, a user
could potentially have multiple VMs, each with access to a different corpus.
However, the handling of such an access permissions system is outside the scope
of this paper.

The fundamental element that provides the guarantees

\subsection{Maintenance Mode}



\section{Implementation}
Suspendisse potenti. Donec ornare metus sem, vel semper arcu fermentum in. Nunc
vehicula eros turpis, ut rhoncus nunc sodales pulvinar. Duis sed est interdum,
semper mi a, cursus sapien. Nulla elementum congue purus, eu faucibus nibh
congue et. Ut diam metus, euismod ac quam in, gravida dictum turpis. Vivamus
hendrerit facilisis tempor. Curabitur hendrerit orci ac libero vestibulum, sit
amet dictum enim viverra. Nulla rutrum dolor ligula, nec rutrum massa suscipit
volutpat. Aliquam id mi at velit aliquet imperdiet at nec mi. Integer in massa
nec augue dapibus ullamcorper. Aliquam nec sollicitudin elit.

\section{Discussion and Future Work}
\label{sec:Discussion}
In eu gravida ligula, sit amet tincidunt nisi. Interdum et malesuada fames ac
ante ipsum primis in faucibus.  Vestibulum venenatis non sem vitae hendrerit.
Cras accumsan vel mauris ut vestibulum. Pellentesque auctor quam ut nisl
commodo malesuada. Phasellus adipiscing placerat tortor, in convallis massa
cursus eu. Vestibulum ullamcorper lorem vel nunc dictum, elementum pretium
nulla tincidunt.

\section{Conclusions}

Interdum et malesuada fames ac ante ipsum primis in faucibus. Nulla facilisi.
Etiam viverra condimentum accumsan. Mauris sed dolor sed mi dignissim
vulputate. Duis \cite{clark:pct} pharetra purus purus, eu ultricies ligula dignissim vel. Sed
nec mauris a turpis luctus placerat id ut ipsum. Fusce quis tincidunt dolor.
Nunc mattis ornare sem, vel consectetur ipsum iaculis ac. Aenean nulla lacus,
iaculis sed tempor nec, euismod suscipit dui.

% References
\bibliographystyle{abbrv}
\bibliography{paper}

%\balancecolumns
\end{document}
